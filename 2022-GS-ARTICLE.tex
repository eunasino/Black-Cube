%!TEX TS-program = xelatex
%!TEX encoding = UTF-8 Unicode
%!TEX root = 2022-GS-ARTICLE.tex
%----------------------------------------------------------------- LANGUAGES ---
\newcommand{\mylanguages}{italian} % in reverse order
%---------------------------------------------------------- TITLE & SUBTITLE ---
\newcommand{\mytitle}{Black Cube}
\newcommand{\mysubtitle}{Denaturalizzazione estremizzata, alienazione dalla dimensione umana}
%----------------------------------------------------------------- AUTHOR(s) ---
\newcommand{\authorone}{Giancarlo Bottalico}
\newcommand{\institutione}{Conservatorio di musica "N. Piccinni", Bari}
\newcommand{\emailone}{giancarlobottalico@gmail.com}
%-------------------------------------------------------------------------------
%-------------------------------------------------------------- STYLE GS2020 ---
\input{gs2022.tex}
%------------------------------------------------------------ BEGIN DOCUMENT ---
\begin{document}
	\maketitle
	\thispagestyle{empty}
	%-------------------------------------------------------------------- ABSTRACT -
	% The abstract is an external txt file inside the includes folder
	%-------------------------------------------------------------------------------
	%the "*" semanthic wrote before title serve to not give an item number to each section or subsection
	
\section{CONTESTO}
Noi esseri umani riceviamo e restituiamo energia alla nostra dimensione natia. Diamo il nostro contributo ad una catena di eventi e fenomeni che muovono e formano la nostra dimensione, ciascuno di questi fattori è determinante al fine del mantenimento dell'equilibrio di questa struttura. Un fenomeno che, in questo periodo storico, muove e influenza in maniera sempre più importante le azioni degli esseri umani, è la \textbf{ludicizzazione} (o gamification).

\subsection{Introduzione}
Il fenomeno della gamification prende luogo nel momento in cui l'inconscio di un essere umano viene influenzato da azioni e meccanismi appartenenti ad una dimensione virtuale creata da noi stessi: quella dei \textbf{videogiochi}. Questo porta inevitabilmente all'insidiamento, nelle profondità emotive del giocatore, di determinati elementi della dimensione dei videogiochi, trasportati poi nella nostra dallo stesso utente.
Il concetto di ludicizzazione ha preso piede a partire dalla conferenza, tenutasi nel febbraio 2010 in occasione del "D.I.C.E. Summit" di Las Vegas, dal game designer \textbf{Jesse Schell}.
La gamification non è solo un fenomeno alieno alla nostra dimensione che si insidia sempre più nelle nostre realtà, ma rappresenta anche uno strumento estremamente potente ed efficace: si possono infatti veicolare messaggi di qualsiasi tipo, persino al fine di indurre l'utente a compiere determinati comportamenti, permettendo di raggiungere specifici obiettivi, personali o d’impresa. Al centro di questo approccio va sempre collocato l’utente ed il suo coinvolgimento attivo basato sull'interazione e sul divertimento, da qui il termine \textbf{ludicizzazione}.
Ma in che modo le azioni svolte all'interno di un videogioco influenzano quelle dei loro stessi creatori: gli esseri umani? E soprattutto quali sono le dinamiche che vengono trasportate da una dimensione ad un'altra? Ecco le principali:

\begin{compactitem}
	\item Finire un livello per raggiungerne un'altro
	\item Esibire distintivi e premi
	\item Accumulare punti
	\item Ottenere ricompense e doni
	\item individuare necessariamente dei nemici
\end{compactitem}


	\begin{figure}[h]
		\begin{center}
		\includegraphics[width=.40\textwidth]{img/image7.jpg}
		\label{gr01}
		\end{center}
	\end{figure}

\section{L'IDEA}
Immagina una dimensione tanto \textbf{lontana} da avere leggi fisiche a noi incomprensibili, ma abbastanza \textbf{vicina} da poterci interagire direttamente. Immagina una dimensione nella quale è possibile tutto ciò che non lo è nella nostra dimensione natia. É estremamente complesso anche solo da idealizzare? Eppure si tratta di una \textbf{nostra creazione}, ma nessuno ha idea di come si muovano gli oggetti al suo interno.
	
	\subsection{Genesi}
	Siamo legati direttamente ad una sola dimensione, ovvero l'unica realtà che ci rende vivi e ci da forma. Ma con il progredire degli anni l'umanità sta smarrendo la concezione di realtà, creando dimensioni virtuali che stanno ingoiando la nostra natura.
	La \textbf{denaturalizzazione} è un fenomeno artificiale che porta a smontare le componenti di un oggetto per poterle riformulare singolarmente. Questo strumento è un tassello fondamentale nello sviluppo di realtà virtuali come i \textbf{videogiochi}: dimensioni fisiche non reali.
	Senza aver compreso la dimensione degli eventi reali, stiamo azzardando ad immergerci in realtà espanse alternative a quella nella quale viviamo, che non abbiamo ancora indagato a pieno.
	
	\begin{figure}[h] %il comando [h] serve a posizionare la figura in questo punto specifico del documento
		\begin{quote}
			\textit{Così, guidato dagli algoritmi, l'essere umano perde sempre più il proprio potere di agire, la propria autonomia. Si trova dinanzi ad un mondo che sfugge alla sua comprensione. Si attiene a decisioni algoritmiche che non riesce a capire fino in fondo, gli algoritmi diventano scatole nere}
		\end{quote}
		\caption{\textbf{Byung - Chul Han, "Le non cose", 2022}}
	\end{figure}
	
	Ma \textbf{Black Cube} va persino oltre tutto questo. L'opera è un'estremizzazione del concetto di denaturalizzazione: l'obiettivo è creare una terza realtà fisica alternativa nella quale persino le nostre realtà aumentate vengono denaturalizzate e portate ad un livello più profondo di virtualità e di denaturalizzazione.
	Con quest'opera intendo dunque creare una realtà alternativa alla realtà alternativa denaturalizzandone ulteriormente l'astrazione, nella quale ogni gesto funzionale alle sue leggi fisiche diventa un gesto compositivo: l'unico linguaggio in grado di mettere in comunicazione questa terza dimensione difficilmente accessibile con la nostra concreta realtà è la \textbf{musica}.
	L'opera infatti è un'installazione musicale interattiva che attribuisce al giocatore la facoltà di \textbf{compositore} e \textbf{fruitore} della stessa, senza che neanche possa rendersene conto.
	L'intento è evidenziare lo smarrimento al quale andiamo in contro, l'allontanamento dalla nostra dimensione natia, di conseguenza fornire uno strumento per comprendere ciò che non controlliamo, per dettare le regole del gioco: la musica.
	
	\subsection{Interazione}
	Snocciolando questa terza dimensione, naturalmente emergono dei quesiti, tra i quali: in che modo questo videogioco gestisce l'estremizzazione della denaturalizzazione? In cosa consiste questa dimensione espansa? Come può il fruitore dell'opera diventare il compositore della musica che ne detta le regole fisiche? In che modo  l'opera rappresenta l'allontanamento dalla nostra reale dimensione?
	L'essenza dell'opera sta nell'\textbf{implementazione dell'audio}: il sound design è realizzato in maniera alternativa rispetto a quella tradizionale, creando suoni che interagiscano con i gesti del giocatore per astrazione più che per causalità. Gli eventi sono denaturalizzati: distaccano il gesto dell'oggetto sonoro dal suono riprodotto.
	Si tratta di un gioco di \textbf{aspettative}: al compimento di un gesto vengono generati dei suoni elettroacustici che "normalmente" non corrispondono ai suoni generati in un videogioco (quindi nella seconda dimensione: la realtà virtuale). Si creano dunque impulsi nervosi e sfuggenti dai gesti morbidi e dilatati, oppure suoni con lungo tempo di attacco e di decadimento dai gesti semplici e rapidi: i suoni e la musica non soddisfano dunque le aspettative del giocatore, perchè sembrano rispondere a delle leggi fisiche a noi sconosciute, appartenenti ad una realtà alternativa persino a quella virtuale.
	Un altro fattore risultante del gioco è il fine: il gioco non ha una missione, un obiettivo o un boss finale come nella "tradizionale" realtà virtuale: questo fattore simboleggia alla perfezione lo smarrimento, la mancanza di una meta o di una rotta da seguire, l'incomprensione della dimensione nella quale ci si trova
	
	\begin{figure}[h]
		\begin{center}
			\includegraphics[width=0.5\textwidth]{img/animation.jpg}
			\caption {\textbf{Frame della visuale del main character all'interno di una delle scene di Black Cube}}
		\end{center}
	\end{figure}

\section{L'OPERA}
Nei seguenti paragrafi entrerò nel merito della realizzazione concreta dell'opera, descrivendo ogni passaggio volto alla creazione della creazione dello spazio \textbf{fisico} e \textbf{virtuale} di Black Cube

	\subsection{Spazio Virtuale}
	Gli script di game engine di Black Cube sono scritti in C++. I soundbank per l'implementazione audio sono generati dal middleware Wwise e vengono caricati nell'architettura del gioco tramite uno script chiamato "sound.cpp". Di seguito la gerarchia degli eventi e delle azioni del gioco. Ciascuna con un proprio trigger che richiama un file audio, i quali vengono interpolati randomicamente ad ogni trigger in più parametri quali: pitch, gain, Eq(LowPass filter e HighPass filter), Flanging(Feedback, dry/wet). In modo tale che suonino sempre diversamente, rendendo l'esperienza esclusiva ad ogni singola azione del giocatore.
	Come si può vedere nella figura 2, gli eventi all'interno del soundbank "Main" sono organizzati in diverse sezioni, quali:
	\begin{compactitem}
		\item Items: suoni associati agli eventi delle diverse scene
		\item Magic: suoni associati alle armi magiche del main character
		\item Main Character: suoni associati alle azioni fisiche del giocatore
		\item Monsters: suoni associati alle azioni fisiche dei nemici
	\end{compactitem}

\newpage
	
	\begin{figure}[h]
		\begin{center}
			\includegraphics[width=.47\textwidth]{img/image2.jpg}
			\caption{\textbf{Gerarchia degli eventi all'interno dei soundbanks "Main" e "Maps"}}
			\label{gr01}
		\end{center}
	\end{figure}

	\begin{figure}[h]
		\begin{center}
			\includegraphics[width=.47\textwidth]{img/image3.jpg}
			\caption{\textbf{Gerarchia degli eventi sonori delle scene}}
			\label{gr01}
		\end{center}
	\end{figure}
	
	\begin{figure}[h]
		\begin{center}
			\includegraphics[width=.47\textwidth]{img/image4.jpg}
			\caption{\textbf{Gerarchia degli eventi sonori delle armi del giocatore}}
			\label{gr01}
		\end{center}
	\end{figure}
		
	\begin{figure}[h]
		\begin{center}
			\includegraphics[width=.47\textwidth]{img/image5.jpg}
			\caption{\textbf{Gerarchia degli eventi sonori delle azioni fisiche del giocatore}}
			\label{gr01}
		\end{center}
	\end{figure}
	
	\begin{figure}[h]
		\begin{center}
			\includegraphics[width=.47\textwidth]{img/image6.jpg}
			\caption{\textbf{Gerarchia degli eventi sonori delle azioni fisiche dei nemici}}
				\label{gr01}
		\end{center}
	\end{figure}
	
	Si può notare che sono presenti diverse icone grafiche associate a ciascun evento. Questi simboli indicano il tipo di evento che viene richiamato per l'azione compiuta nel gioco. I principali sono:
	
	\begin{compactitem}
		\item Simple Event: questo è il più semplice trigger utilizzabile in Wwise. Rappresenta un evento audio che viene riprodotto una volta richiamato.
		\item Random Container: un insieme di eventi audio che vengono richiamati in maniera randomica ad ogni trigger. utilizzato ad esempio per i footsteps o per i jump sounds
		\item Switch Container: Utilizzato quando si desidera selezionare diversi eventi audio in base ad uno switch. il quale potrebbe corrispondere a una situazione di gioco specifica: uno stato.
		\item Blend Container: utilizzato per mescolare più eventi audio insieme, utile nelle transizioni. Utilizzato per alcune armi, laddove il loro utilizzo implica il trigger del lancio del proiettile e quello del suo impatto
		\item Sequence Container: Rriproduce gli eventi audio contenuti in sequenza, uno dopo l'altro, senza interruzioni.
	\end{compactitem}
	
	Alcuni di questi eventi sono modulati tramite \textbf{RTPC} (Real-Time Parameter Control). Si tratta di un sistema che consente di controllare i parametri sonori in tempo reale durante la riproduzione del gioco. 
	L'RTPC permette di modificare i valori di un parametro, come il volume o la frequenza di taglio di un filtro, in modo dinamico e in base a determinati eventi o situazioni all'interno del gioco. Ad esempio, si potrebbe utilizzare un RTPC per aumentare il volume del gesto sonoro quando il giocatore si avvicina a una fonte di rumore, o per modificare la velocità di riproduzione della musica in base alla velocità di movimento del personaggio. In questo modo, è possibile creare un'esperienza sonora più immersiva e realistica per il giocatore.
	
	Il formato audio nel quale è stato missato il gioco è stereo a due canali. Nel calcolatore viene effettuato un routing interno dei segnali audio: l'audio in uscita dal videogioco viene mandato al canale virtuale del mixer Audient 1/2, il cui master send verrà disattivato in modo tale che il suo audio non esca dal master, il canale in questione viene impostato come fonte di Loopback. In un codice associato scritto in Maxmsp, l'audio viene prelevato dai canali 1/2 del mixer virtuale e processato secondo dei parametri descritti alla fine del paragrafo. L'audio in uscita di MaxMsp viene mandato ai canali 3/4 del mixer virtuale Audient, i quali vanno direttamente in uscita dal master. In questo modo ascolteremo solo l'audio in uscita da MaxMsp, che viene a sua volta prelevato da Black Cube.
	Il codice di MaxMsp ha lo scopo di restituire un feedback organico che si contrapponga all'astrazione: durante l'esperienza del giocatore, all'aumentare della sua immersione nella dimensione di Black Cube, dei suoni legati alla nostra dimensione, simboli di naturalezza e quotidianità, vengono richiamati dalla patch ammassandosi sempre di più e aumentando il loro volume con il passare del tempo.
	In oltre, l'audio in uscita da Black Cube viene filtrato con un LowPassFilter, la cui frequenza di taglio si abbassa verso lo 0 all'aumentare del tempo.
	Il soundscape generato dalla totalità dell'opera veicola la crescente necessità di \textbf{ritorno} alla nostra dimensione natia in seguito all'alienazione da questa data dall'esperienza immersiva in questo livello profondo di astrazione e virtualità. Maggiormente ci si smarrisce nella terza dimensione con il passare del tempo, maggiore sarà la necessità istintiva di evaderne per fare ritorno alla realtà.

\newpage

	\begin{figure}[h]
			\hspace{1cm}\includegraphics[width=.9\textwidth]{img/image8.jpg}
			\caption{\textbf{codice scritto in MaxMsp}}
			\label{gr01}
	\end{figure}
	
	\subsection{Spazio Fisico}
	Ora estendiamo questa terza dimensione in uno spazio fisico appartenente alla nostra realtà. L'immersione in Black Cube prevede che il giocatore entri in una stanza rettangolare senza finestre, illuminata solo dalla luce dello schermo principale sul quale è proiettato Black Cube, e dagli schermi laterali sui quali sono proiettate immagini iconiche di naturalezza e quotidianità appartenenti alla nostra dimensione. La luminosità degli schermi laterali aumenta al trascorrere del tempo.
	Posto al di sotto dello schermo principale, in linea con le spalle del fruitore, uno specchio riflette la sua immagine. Con il passare del tempo, dunque all'aumento del livello di immersione del giocatore nella dimensione di Black Cube, La superficie dello specchio si scurisce, opacizzando lentamente l'immagine riflessa. Lo specchio torna a riflettere normalmente la luce solo una volta giunti al termine dell'esperienza
	Il fine è di suggestionare la necessità crescente di evasione dalla terza dimensione in base al livello di immersione e profondità nella stessa. L'esperienza è una suggestione per il fruitore al voler tornare al mondo reale, stimolata da diversi elementi dell'opera:
	\begin{compactitem}
		\item Le immagini proiettate sugli schermi laterali, che simboleggiano la perdita degli elementi naturali appartenenti alla nostra dimensione
		\item Il proprio riflesso nello specchio che scompare lentamente simboleggiando lo smarrimento della naturale immagine di se stessi e l'allontanamento dalle leggi fisiche del nostro mondo
		\item I richiami sonori esasperati di oggetti appartenenti alla nostra dimensione, che comunicano la necessità di farne ritorno
	\end{compactitem}

\newpage
-
\vspace{8cm}

	\begin{figure}[h]
		\begin{center}
			\includegraphics[width=6cm]{img/image1.png}
			\caption{\textbf{Modello 3D dello spazio fisico per l'allestimento dell'installazione}}
			\label{gr01}
		\end{center}
	\end{figure}

L'installazione è realizzata con un'approccio volto all'aumento dell'immersione del giocatore. L'obiettivo è fornire un'esperienza a 360° in grado di coinvolgere totalmente il fruitore dell'opera. Maggiore è il suo livello di immersione in Black Cube, maggiore è la comprensione dell'essenza dell'opera, il coinvolgimento emotivo e la sensibilizzazione che questa può dare al giocatore.

\section{SVILUPPI ED APPLICAZIONI}
Tutto ciò trasforma Black Cube in un opera immersiva, coinvolgente, tattile. Ma che tipo di stimoli restituisce l'esperienza che offre? Cosa c'è da scoprire in un livello così profondo di astrazione e virtualtà? può Black Cube essere un'approccio suggestivo applicabile alla realtà? Può in qualche modo aumentare l'immersione e dunque il coinvolgimento del giocatore?

\subsection{Gamification}
Chiudiamo il cerchio. L'intento dell'opera è di accompagnare il fruitore attraverso un'esperienza interattiva lineare che passa per diverse fasi:

\begin{enumerate}
	\item Totale immersione nella terza dimensione: quella di Black Cube
	\item Percezione della crescente necessità di evasione dalla stessa e di ritorno a quella natia
	\item ludicizzazione: comprensione da parte dell'utente di aver insidiato dentro di se meccanismi alieni alla sua dimensione, e di averceli portati
\end{enumerate}

Mi sono educato ad educare. Al veicolare emozioni ed insegnamenti. Al dare un perchè alle mie vocazioni artistiche. Ad esprimere il mio opus artistico in opere che non siano fini a se stesse. L'esperienza da me offerta deve lasciare qualcosa al fruitore, deve accompagnarlo in un viaggio suggestivo che lo ispiri a riflettere su determinate dinamiche.
La sensibilizzazione alla quale intendo sottoporre gli utenti è utile al fine di comprendere che stiamo cercando di controllare dimensioni a noi aliene, e che spesso ci si può intossicare in questo intento, portando con se meccanismi tossici ed estranei influenzando così l'equilibrio naturale della nostra dimensione.
	
	\begin{figure}[h]
		\begin{center}
			\includegraphics[width=.47\textwidth]{img/image9.jpg}
			\label{gr01}
		\end{center}
	\end{figure}

	
	
	\vfill\null
	
	\newpage % USE NEWPAGE TO FORCE COLUMNN INTERRUPTION
	%-------------------------------------------------------------------------------
	%-------------------------------------------------------------------------------
	
	%--------------------------------------------
	%----------------larghezza massima del codice
	
	\vfill\null
	
	\raggedright
	%\bibliographystyle{unsrt}
	%\printbibliography
	
\end{document}

%%%%%%%%%%%%%%%%%%%%%%%%%%%%%%%%%%%%%%%%%%%%%%%%%%%%%%%%%%%%%%%%%%%%%%%%%%%%%%%%
% 2020 GIUSEPPE SILVI ARTICLE TEMPLATE BASED ON
%%%%%%%%%%%%%%%%%%%%%%%%%%%%%%%%%%%%%%%%%%%%%%%%%%%%%%%%%%%%%%%%%%%%%%%%%%%%%%%%
% Journal Article
% LaTeX Template
% Version 1.4 (15/5/16)
% This template has been downloaded from:
% http://www.LaTeXTemplates.com
% Original author:
% Frits Wenneker (http://www.howtotex.com) with extensive modifications by
% Vel (vel@LaTeXTemplates.com)
% License:
% CC BY-NC-SA 3.0 (http://creativecommons.org/licenses/by-nc-sa/3.0/)
%%%%%%%%%%%%%%%%%%%%%%%%%%%%%%%%%%%%%%%%%%%%%%%%%%%%%%%%%%%%%%%%%%%%%%%%%%%%%%%%
